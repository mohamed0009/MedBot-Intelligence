\documentclass[12pt,a4paper]{report}
\usepackage[utf8]{inputenc}
\usepackage[french]{babel}
\usepackage[T1]{fontenc}
\usepackage{geometry}
\usepackage{titlesec}
\usepackage{tocloft}
\usepackage{graphicx}
\usepackage{hyperref}
\usepackage{fancyhdr}
\usepackage{enumitem}
\usepackage{float}
\usepackage{xcolor}
\usepackage{setspace}

% Configuration de la page
\geometry{
    left=3cm,
    right=2.5cm,
    top=2.5cm,
    bottom=2.5cm
}

% Configuration des en-têtes et pieds de page
\pagestyle{fancy}
\fancyhf{}
\fancyhead[L]{\leftmark}
\fancyhead[R]{}
\fancyfoot[C]{\thepage}

% Configuration du style plain pour numérotation en bas
\fancypagestyle{plain}{%
\fancyhf{}%
\fancyfoot[C]{\thepage}%
\renewcommand{\headrulewidth}{0pt}%
}

% Configuration des titres de chapitres en vert
\titleformat{\chapter}[display]
{\normalfont\huge\bfseries\color{green!70!black}}{\chaptertitlename\ \thechapter}{20pt}{\Huge\color{green!70!black}}

% Configuration des titres de sections en vert
\titleformat{\section}
{\normalfont\Large\bfseries\color{green!70!black}}{\thesection}{1em}{}

% Configuration des sections non numérotées en vert
\titleformat{name=\section,numberless}
{\normalfont\Large\bfseries\color{green!70!black}}{}{0em}{}

% Configuration de la table des matières avec titres en vert
\renewcommand{\cftchapfont}{\color{green!70!black}\bfseries}
\renewcommand{\cftsecfont}{\color{green!70!black}\bfseries}
\renewcommand{\cftsubsecfont}{\normalfont}

% Configuration des liens hypertexte
\hypersetup{
    colorlinks=true,
    linkcolor=black,
    filecolor=black,
    urlcolor=blue,
    citecolor=black
}

% Numérotation des pages pour les pages préliminaires en chiffres romains
\pagenumbering{roman}

\begin{document}

% Interligne simple pour les pages préliminaires
\singlespacing

% Page de garde
\newpage
\thispagestyle{empty}
\pagenumbering{roman}
\setcounter{page}{1}

% En-tête avec logo EMSI
\begin{minipage}{0.5\textwidth}
\begin{flushleft}
% Logo EMSI (à ajouter)
%\includegraphics[width=3cm]{images/logo_emsi.png}\\
\textbf{EMSI}\\
\textit{ÉCOLE MAROCAINE DES SCIENCES DE L'INGENIEUR}\\
\textit{Membre de HONORIS UNITED UNIVERSITIES}
\end{flushleft}
\end{minipage}

\vspace{2.5cm}

% Titre principal
\begin{center}
{\Huge\bfseries Cahier des charges}
\end{center}

\vspace{2.5cm}

% Introduction du thème
\begin{center}
Projet sous le thème de\\
\vspace{0.3cm}
\noindent\rule{0.7\textwidth}{0.5pt}
\end{center}

\vspace{1cm}

% Thème du projet
\begin{center}
{\Large\bfseries Formation professionnelle\\
interactive avec IA générative (coach\\
virtuel)}
\vspace{0.3cm}\\
\noindent\rule{0.7\textwidth}{0.5pt}
\end{center}

\vspace{4cm}

% Informations de l'équipe et date
\begin{minipage}{0.5\textwidth}
\begin{flushleft}
\textbf{Réalisé par l'équipe de :}\\
\vspace{0.5cm}
\begin{itemize}[leftmargin=*]
    \item BAANNI Zaineb
    \item IDRISSI Mohamed
    \item OUGAS Fadoua
\end{itemize}
\end{flushleft}
\end{minipage}
\begin{minipage}{0.5\textwidth}
\begin{flushright}
\textbf{Rendu le}\\
\vspace{0.5cm}
07/11/2025
\end{flushright}
\end{minipage}

\vspace{3cm}

% Professeur
\begin{center}
\textbf{Professeur : Mme. BOUSQAOUI Halima}
\end{center}

\vspace*{\fill}

\newpage

% Page blanche
\newpage
\thispagestyle{empty}
\mbox{}

% Dédicace
\newpage
\thispagestyle{plain}
\vspace*{2cm}
\begin{center}
{\Huge\bfseries\color{green!70!black} Dédicace}
\end{center}

\vspace{2cm}

\begin{center}
\textbf{\textit{Avec un énorme plaisir, un cœur ouvert et une immense joie, je dédie mon travail :}}
\end{center}

\vspace{2.5cm}

\begin{center}
\large
\singlespacing

{\Large\bfseries\color{green!70!black}À mes chers parents,}

\vspace{1cm}

Ce modeste travail est le reflet de vos espoirs, le fruit de vos sacrifices silencieux et de votre soutien inébranlable.\\
Qu'il soit l'expression de ma profonde gratitude et de mon amour éternel.

\vspace{2cm}

{\Large\bfseries\color{green!70!black}À mes amis les plus proches,}

\vspace{1cm}

Merci pour votre présence constante, votre fidélité sincère et vos encouragements qui ont su éclairer les moments de doute.

\vspace{2cm}

{\Large\bfseries\color{green!70!black}À mes professeurs,}

\vspace{1cm}

Je vous exprime ma reconnaissance pour la qualité de votre enseignement, votre disponibilité et vos précieux conseils, sans lesquels ce projet n'aurait pu aboutir dans sa forme actuelle.

\end{center}

\vspace*{\fill}
\newpage

% Remerciement
\newpage
\thispagestyle{plain}
\vspace*{2cm}
\begin{center}
{\Huge\bfseries\color{green!70!black} REMERCIEMENTS}
\end{center}

\vspace{2cm}

\begin{center}
\large
\onehalfspacing

En préambule à ce projet, nous souhaitons exprimer notre profonde reconnaissance à toutes les personnes qui, de près ou de loin, ont contribué à l'aboutissement de ce travail ainsi qu'à la réussite de notre formation.

\vspace{1.5cm}

Nous tenons à remercier tout particulièrement \textbf{Madame BOUSQAOUI Halima} pour ses enseignements en JEE et pour son encadrement tout au long de ce projet.

\vspace{1.5cm}

Nos remerciements vont également à \textbf{Madame SBAI Hanae} pour ses précieux enseignements en Flutter qui nous ont permis de développer l'interface de notre application.

\vspace{1.5cm}

Nous exprimons notre gratitude à \textbf{Monsieur ESSABBAR Driss} pour ses enseignements en management de qualité et pour ses conseils précieux.

\vspace{1.5cm}

Que toutes les personnes ayant contribué, de près ou de loin, à la réussite de ce projet trouvent ici l'expression de notre sincère reconnaissance.

\end{center}

\vspace*{\fill}
\newpage

% Résumé
\newpage
\thispagestyle{plain}
\vspace*{2cm}
\begin{center}
{\Huge\bfseries\color{green!70!black} Résumé}
\end{center}
\vspace{1.5cm}

\begin{center}
\large
\onehalfspacing

Ce projet présente le développement d'une plateforme de formation professionnelle interactive intégrant un coach virtuel basé sur l'intelligence artificielle générative. La solution proposée répond aux limitations de la formation traditionnelle en offrant une expérience d'apprentissage personnalisée, dynamique et accessible. L'architecture technique combine Flutter pour le front-end, Spring Boot pour le back-end, et un module d'IA (LLM + RAG) pour le coach virtuel. Le système permet la génération automatique de contenu pédagogique, la personnalisation des parcours d'apprentissage et un suivi efficace par les formateurs. Le développement suit une méthodologie agile SCRUM sur une période de 8 semaines, organisée en sprints itératifs permettant une validation progressive des fonctionnalités.

\end{center}

\newpage

% Abstract
\newpage
\thispagestyle{plain}
\vspace*{2cm}
\begin{center}
{\Huge\bfseries\color{green!70!black} Abstract}
\end{center}
\vspace{1.5cm}

\begin{center}
\large
\onehalfspacing

This project presents the development of an interactive professional training platform integrating a virtual coach based on generative artificial intelligence. The proposed solution addresses the limitations of traditional training by offering a personalized, dynamic, and accessible learning experience. The technical architecture combines Flutter for the front-end, Spring Boot for the back-end, and an AI module (LLM + RAG) for the virtual coach. The system enables automatic generation of educational content, personalization of learning paths, and effective monitoring by trainers. Development follows an agile SCRUM methodology over an 8-week period, organized into iterative sprints allowing progressive validation of features.

\end{center}

\newpage

% Liste des figures
\newpage
\thispagestyle{empty}
\listoffigures
\addcontentsline{toc}{chapter}{Liste des figures}
\newpage

% Glossaire
\newpage
\thispagestyle{plain}
\vspace*{2cm}
\begin{center}
{\Huge\bfseries\color{green!70!black} Glossaire}
\end{center}
\vspace{1.5cm}

\begin{description}
    \item[LLM] Large Language Model - Modèle de langage de grande taille
    \item[RAG] Retrieval-Augmented Generation - Génération augmentée par récupération
    \item[API REST] Application Programming Interface - Representational State Transfer
    \item[SCRUM] Méthodologie agile de gestion de projet
    \item[UML] Unified Modeling Language - Langage de modélisation unifié
    \item[Microservices] Architecture logicielle basée sur des services indépendants
\end{description}

\newpage

% Table des matières
\newpage
\thispagestyle{plain}
\vspace*{1cm}
\begin{center}
{\Huge\bfseries\color{green!70!black} Table des Matières}
\end{center}
\vspace{1cm}
\singlespacing
\tableofcontents
\onehalfspacing
\newpage

% Début du contenu principal avec numérotation arabe
\pagenumbering{arabic}
\setcounter{page}{1}

% Configuration de l'interligne 1.5 pour le contenu principal
\onehalfspacing

% Introduction générale
\chapter*{\color{green!70!black} Introduction générale}
\addcontentsline{toc}{chapter}{Introduction générale}

\section*{Introduction globale}
\addcontentsline{toc}{section}{Introduction globale}

La formation professionnelle connaît actuellement une transformation majeure, portée par les avancées technologiques et notamment l'intelligence artificielle générative. Ce projet s'inscrit dans cette dynamique en proposant le développement d'une plateforme innovante de formation professionnelle interactive, intégrant un coach virtuel capable d'adapter l'apprentissage aux besoins spécifiques de chaque apprenant.

Ce rapport présente l'ensemble du processus de développement, depuis l'analyse des besoins jusqu'à l'implémentation finale, en passant par la conception et le choix des technologies. Il est structuré en quatre chapitres principaux, chacun abordant un aspect spécifique du projet.

\section*{Présentation des chapitres}
\addcontentsline{toc}{section}{Présentation des chapitres}

\textbf{Chapitre 1 : Contexte général du projet}

Ce premier chapitre présente le contexte et la problématique du projet, ainsi que les objectifs visés et la solution proposée. Il détaille également la démarche méthodologique adoptée, notamment l'utilisation de la méthode agile SCRUM, l'organisation de l'équipe et la planification du travail à travers les différents sprints sur une période de 8 semaines.

\textbf{Chapitre 2 : Analyse et conception}

Le deuxième chapitre expose l'analyse et la conception du système à travers la modélisation UML. Il présente les diagrammes de cas d'utilisation, les diagrammes de séquence pour les principales fonctionnalités, ainsi que le diagramme de classes définissant la structure statique du système. Cette modélisation constitue la base de référence pour la phase d'implémentation.

\textbf{Chapitre 3 : Technologies et outils utilisés}

Le troisième chapitre détaille l'ensemble des technologies, frameworks et outils choisis pour le développement de la plateforme. Il présente les technologies front-end (Flutter), back-end (Spring Boot), le module d'intelligence artificielle (LLM + RAG), ainsi que les outils de base de données, de développement, de qualité et de gestion de projet.

\textbf{Chapitre 4 : Implémentation et mise en œuvre}

Le quatrième et dernier chapitre décrit l'implémentation concrète de la plateforme, en présentant l'architecture logicielle mise en place, les différentes interfaces développées (web et mobile) et les choix techniques effectués lors de la réalisation. Il démontre la concrétisation des objectifs définis dans le cahier des charges.

\newpage

% CHAPITRE 1
\chapter{CONTEXTE GENERALE DU PROJET}
\setcounter{page}{2}

\section{Introduction}

Dans un contexte où la formation professionnelle évolue rapidement vers des approches plus personnalisées et interactives, l'intégration de l'intelligence artificielle générative représente une opportunité majeure pour transformer l'expérience d'apprentissage. Ce chapitre présente le contexte général du projet de développement d'une plateforme de formation professionnelle interactive avec un coach virtuel basé sur l'IA générative.

Nous commencerons par présenter le projet dans son ensemble, en identifiant la problématique à laquelle il répond, les objectifs poursuivis et la solution proposée. Ensuite, nous détaillerons la démarche méthodologique adoptée, notamment l'utilisation de la méthode agile SCRUM, ainsi que l'organisation de l'équipe et la planification du travail à travers les différents sprints.

\section{Présentation du projet}

\subsection{Problématique}

La formation professionnelle traditionnelle rencontre plusieurs limitations significatives qui impactent l'efficacité de l'apprentissage et la satisfaction des apprenants. Parmi ces défis, on peut identifier :

\begin{itemize}
    \item \textbf{Manque de personnalisation} : Les formations traditionnelles adoptent souvent une approche uniforme qui ne tient pas compte des différences individuelles en termes de niveau, de rythme d'apprentissage et de besoins spécifiques de chaque apprenant.
    
    \item \textbf{Limitation de l'interactivité} : L'interaction entre le formateur et l'apprenant est souvent limitée par les contraintes de temps et le nombre d'apprenants, ce qui réduit les opportunités de questions, d'explications supplémentaires et de feedback personnalisé.
    
    \item \textbf{Suivi individualisé difficile} : Le suivi de la progression de chaque apprenant et l'adaptation du parcours en fonction des performances représentent un défi majeur pour les formateurs, particulièrement dans des contextes avec un grand nombre d'apprenants.
    
    \item \textbf{Accès limité aux ressources adaptées} : Les apprenants ont souvent du mal à trouver des ressources pédagogiques adaptées à leur niveau et à leurs besoins spécifiques, ce qui peut ralentir leur progression.
    
    \item \textbf{Manque de flexibilité temporelle} : Les formations traditionnelles imposent souvent des horaires fixes qui ne correspondent pas toujours aux disponibilités des apprenants.
\end{itemize}

Ces limitations soulignent la nécessité de développer des solutions innovantes capables d'offrir une expérience d'apprentissage plus dynamique, personnalisée et accessible. L'intégration de l'intelligence artificielle générative constitue une réponse prometteuse à ces défis, permettant de créer un coach virtuel capable d'interagir de manière naturelle avec les apprenants et d'adapter le contenu pédagogique à leurs besoins spécifiques.

\subsection{Objectifs du projet}

L'objectif principal de ce projet est de développer une plateforme de formation professionnelle interactive intégrant un coach virtuel basé sur l'intelligence artificielle générative. Cette plateforme vise à offrir une expérience d'apprentissage personnalisée, dynamique et accessible à tous les apprenants.

Les objectifs spécifiques du projet sont les suivants :

\begin{enumerate}
    \item \textbf{Développer un système de dialogue intelligent} : Créer un coach virtuel capable de dialoguer en langage naturel avec les apprenants, de comprendre leurs questions et de fournir des réponses adaptées et contextuelles grâce à un modèle d'IA générative (LLM).
    
    \item \textbf{Personnaliser le parcours d'apprentissage} : Mettre en place un système d'analyse des performances qui permet d'adapter automatiquement le parcours de formation en fonction du niveau, du rythme et des besoins de chaque apprenant.
    
    \item \textbf{Générer du contenu pédagogique dynamique} : Développer un module capable de générer automatiquement des exercices, des quiz, des résumés et des études de cas adaptés au domaine de formation et au niveau de l'apprenant.
    
    \item \textbf{Assurer un suivi efficace} : Fournir aux formateurs un tableau de bord complet permettant de suivre la progression des apprenants, d'identifier les difficultés et d'intervenir de manière ciblée.
    
    \item \textbf{Offrir une accessibilité multiplateforme} : Développer une application accessible depuis différents dispositifs (ordinateur, tablette, smartphone) pour garantir une flexibilité maximale aux utilisateurs.
    
    \item \textbf{Maintenir un historique d'apprentissage} : Sauvegarder l'historique des conversations et des activités pour permettre un apprentissage continu et une amélioration progressive du système.
\end{enumerate}

Ces objectifs s'alignent avec les besoins identifiés dans le cahier des charges et visent à transformer l'expérience de formation professionnelle en la rendant plus interactive, personnalisée et efficace.

\subsection{Solution proposée}

Pour répondre à la problématique identifiée et atteindre les objectifs fixés, nous proposons le développement d'une plateforme de formation professionnelle interactive basée sur une architecture moderne et scalable. La solution comprend les composants suivants :

\subsubsection{Architecture technique}

La solution adopte une architecture distribuée basée sur une approche microservices, permettant une séparation claire des responsabilités et une évolutivité optimale :

\begin{itemize}
    \item \textbf{Front-end} : Application hybride développée avec Flutter, permettant un accès multiplateforme (web, iOS, Android) avec une seule base de code. Cette approche garantit une expérience utilisateur cohérente sur tous les dispositifs.
    
    \item \textbf{Back-end} : API REST développée avec Spring Boot, assurant la gestion des utilisateurs, des contenus pédagogiques, de la progression et des interactions avec le système d'IA. Le backend suit une architecture en couches (Controller, Service, Repository) pour une maintenabilité optimale.
    
    \item \textbf{Module d'intelligence artificielle} : Intégration d'un modèle de langage (LLM) combiné avec une approche RAG (Retrieval-Augmented Generation) pour permettre au coach virtuel de générer des réponses contextuelles et pertinentes basées sur le contenu pédagogique disponible. Cette approche garantit la précision et la pertinence des réponses générées.
    
    \item \textbf{Base de données} : Utilisation de MySQL ou PostgreSQL pour stocker les données utilisateurs, les contenus pédagogiques, l'historique des conversations et les statistiques de progression. Le choix de la base de données relationnelle permet une gestion efficace des relations entre les différentes entités.
\end{itemize}

\subsubsection{Fonctionnalités principales}

La solution proposée intègre plusieurs fonctionnalités clés répondant aux besoins identifiés :

\begin{itemize}
    \item \textbf{Gestion des utilisateurs} : Système d'authentification sécurisé avec gestion des rôles (Administrateur, Formateur, Apprenant) et personnalisation des profils. Les utilisateurs peuvent modifier leurs informations et leurs préférences d'apprentissage.
    
    \item \textbf{Interaction avec le coach virtuel} : Interface de chat permettant un dialogue en langage naturel avec le coach virtuel, avec sauvegarde de l'historique des conversations pour un apprentissage continu et contextuel.
    
    \item \textbf{Parcours personnalisé} : Système d'analyse des performances qui adapte automatiquement le contenu et le rythme d'apprentissage selon les besoins de chaque apprenant. Le système recommande des ressources supplémentaires (vidéos, documents, quiz) en fonction de la progression.
    
    \item \textbf{Génération automatique de contenu} : Création dynamique d'exercices, de quiz, de résumés et d'études de cas par l'IA, adaptés au niveau et au domaine de formation. Le contenu est généré en temps réel et peut être validé par les formateurs.
    
    \item \textbf{Tableau de bord} : Interface de suivi pour les formateurs permettant de visualiser les statistiques, la progression et les difficultés des apprenants. Les formateurs peuvent également ajouter, modifier ou valider des contenus pédagogiques.
    
    \item \textbf{Notifications et assistance} : Système de notifications pour rappeler les sessions, les objectifs et fournir un support intégré. Le coach virtuel peut également envoyer des messages de motivation personnalisés.
\end{itemize}

Cette solution offre une approche innovante qui combine les avantages de l'intelligence artificielle générative avec une architecture technique moderne, permettant de répondre efficacement aux défis de la formation professionnelle traditionnelle.

\subsection{Démarche et planification}

Le développement de ce projet suit une approche structurée et méthodique, organisée selon une méthodologie agile pour garantir la flexibilité, la qualité et la réactivité face aux imprévus.

\subsubsection{La méthode SCRUM}

Le projet adopte la méthodologie SCRUM, une approche agile qui privilégie le développement itératif et incrémental. SCRUM organise le travail en cycles courts appelés \textit{sprints}, d'une durée typique d'une à deux semaines, permettant de livrer régulièrement des versions fonctionnelles partielles du produit.

Les principes fondamentaux de SCRUM appliqués dans ce projet sont :

\begin{itemize}
    \item \textbf{Transparence} : Tous les membres de l'équipe et les parties prenantes ont une vision claire de l'état d'avancement du projet grâce aux outils de gestion (Trello, Jira ou Azure DevOps) et aux réunions régulières.
    
    \item \textbf{Inspection} : Des points de contrôle réguliers (daily stand-ups, revues de sprint) permettent d'évaluer l'avancement et d'identifier rapidement les obstacles.
    
    \item \textbf{Adaptation} : La flexibilité de SCRUM permet d'ajuster les priorités et les objectifs en fonction des retours des encadrants et des contraintes techniques rencontrées.
\end{itemize}

Les rituels SCRUM mis en place incluent :

\begin{itemize}
    \item \textbf{Sprint Planning} : Planification des tâches pour chaque sprint avec définition des objectifs et estimation de l'effort. Cette réunion permet de prioriser les fonctionnalités et de s'assurer que l'équipe comprend bien les objectifs.
    
    \item \textbf{Daily Stand-up} : Réunions quotidiennes courtes (15 minutes) pour synchroniser l'équipe et identifier les blocages. Chaque membre présente ce qu'il a fait, ce qu'il va faire et les obstacles rencontrés.
    
    \item \textbf{Sprint Review} : Présentation des fonctionnalités développées aux encadrants pour validation et retours. Cette réunion permet d'ajuster la direction du projet en fonction des retours.
    
    \item \textbf{Sprint Retrospective} : Analyse de ce qui s'est bien passé et des améliorations à apporter pour les prochains sprints. Cette réflexion continue permet d'améliorer les processus de travail.
\end{itemize}

\subsubsection{Pourquoi SCRUM ?}

Le choix de la méthodologie SCRUM pour ce projet se justifie par plusieurs avantages spécifiques :

\begin{enumerate}
    \item \textbf{Adaptabilité aux changements} : Dans un contexte de projet universitaire avec des contraintes de temps et des exigences qui peuvent évoluer, SCRUM permet de s'adapter rapidement aux changements de priorités ou aux retours des encadrants.
    
    \item \textbf{Livraisons régulières} : Les sprints courts permettent de livrer régulièrement des versions fonctionnelles, facilitant la validation progressive par les encadrants et réduisant les risques d'écarts par rapport aux attentes.
    
    \item \textbf{Amélioration continue} : Les rétrospectives régulières permettent d'identifier et de corriger rapidement les problèmes organisationnels ou techniques, améliorant ainsi la qualité du produit final.
    
    \item \textbf{Visibilité et communication} : Les réunions régulières et les outils de gestion assurent une communication fluide entre les membres de l'équipe et avec les encadrants, favorisant l'alignement sur les objectifs.
    
    \item \textbf{Gestion des risques} : L'approche itérative permet d'identifier et de résoudre les problèmes techniques ou fonctionnels dès leur apparition, évitant leur accumulation en fin de projet.
    
    \item \textbf{Motivation de l'équipe} : Les objectifs courts et atteignables de chaque sprint maintiennent la motivation de l'équipe et permettent de célébrer les succès régulièrement.
\end{enumerate}

Cette méthodologie est particulièrement adaptée au développement d'un système complexe intégrant plusieurs technologies (Flutter, Spring Boot, IA) et nécessitant une intégration progressive des différents modules.

\subsubsection{L'équipe et rôles}

L'équipe projet est composée de trois membres, chacun apportant ses compétences spécifiques au développement du système :

\begin{itemize}
    \item \textbf{BAANNI Zaineb} : Membre de l'équipe de développement, contribuant à la conception et à l'implémentation des différents modules du système.
    
    \item \textbf{IDRISSI Mohamed} : Membre de l'équipe de développement, responsable de la mise en œuvre de fonctionnalités spécifiques et de l'intégration des composants.
    
    \item \textbf{OUGAS Fadoua} : Membre de l'équipe de développement, participant à la conception, au développement et à la documentation du projet.
\end{itemize}

Bien que l'équipe soit de petite taille, les rôles SCRUM sont adaptés au contexte :

\begin{itemize}
    \item \textbf{Product Owner} : Représenté par les encadrants (M. ESSABBAR Driss, Mme. SBAI Hanae, Mme. BOUSQAOUI Halima) qui définissent les besoins pédagogiques et fonctionnels et valident les livrables.
    
    \item \textbf{Scrum Master} : Rôle partagé entre les membres de l'équipe, assurant le respect des rituels SCRUM et la résolution des blocages. Chaque membre peut endosser ce rôle selon les besoins.
    
    \item \textbf{Équipe de développement} : Les trois membres de l'équipe travaillent de manière collaborative sur les différentes tâches, avec une répartition flexible selon les compétences et les disponibilités. La collaboration est favorisée par l'utilisation d'outils de gestion de projet et de contrôle de version (GitHub).
\end{itemize}

La communication avec les encadrants se fait à travers des réunions hebdomadaires et des points de contrôle réguliers, permettant de valider les orientations techniques et pédagogiques et d'assurer l'alignement avec les objectifs du projet.

\section{Identification des sprints}

Le projet s'étend sur une durée totale de 8 semaines, organisées en plusieurs sprints selon la méthodologie SCRUM. Chaque sprint a une durée d'une à deux semaines et se concentre sur des objectifs spécifiques. La planification des sprints est la suivante :

\subsection{Sprint 1-2 : Analyse et conception (Semaines 1-2)}

\textbf{Durée} : 2 semaines

\textbf{Objectifs} :
\begin{itemize}
    \item Analyse approfondie des besoins fonctionnels et techniques
    \item Identification des cas d'utilisation et des acteurs
    \item Conception de l'architecture globale du système
    \item Élaboration des diagrammes UML (cas d'utilisation, séquence, classes)
    \item Réalisation des maquettes de l'interface utilisateur
    \item Validation de la faisabilité technique avec les encadrants
\end{itemize}

\textbf{Livrables} :
\begin{itemize}
    \item Cahier des charges détaillé
    \item Diagrammes UML complets
    \item Maquettes de l'interface utilisateur
    \item Document d'architecture technique
\end{itemize}

\subsection{Sprint 3-4 : Développement du backend et base de données (Semaines 3-4)}

\textbf{Durée} : 2 semaines

\textbf{Objectifs} :
\begin{itemize}
    \item Mise en place du projet Spring Boot
    \item Configuration de la base de données (MySQL/PostgreSQL)
    \item Développement des entités et modèles de données
    \item Implémentation des services d'authentification et de gestion des utilisateurs
    \item Création des API REST pour la gestion des profils et des contenus
    \item Tests unitaires des services backend
\end{itemize}

\textbf{Livrables} :
\begin{itemize}
    \item Backend Spring Boot fonctionnel
    \item Schéma de base de données
    \item API REST documentées
    \item Tests unitaires avec couverture minimale
\end{itemize}

\subsection{Sprint 5-6 : Développement frontend et intégration IA (Semaines 5-6)}

\textbf{Durée} : 2 semaines

\textbf{Objectifs} :
\begin{itemize}
    \item Développement de l'application Flutter (web et mobile)
    \item Implémentation des interfaces utilisateur selon les maquettes
    \item Intégration avec les API backend
    \item Développement du module d'interaction avec le coach virtuel
    \item Intégration du module IA (LLM + RAG)
    \item Implémentation de la génération automatique de contenu
    \item Développement du tableau de bord pour les formateurs
\end{itemize}

\textbf{Livrables} :
\begin{itemize}
    \item Application Flutter fonctionnelle
    \item Module d'IA intégré et opérationnel
    \item Interfaces utilisateur complètes
    \item Communication frontend-backend établie
\end{itemize}

\subsection{Sprint 7 : Intégration et tests (Semaine 7)}

\textbf{Durée} : 1 semaine

\textbf{Objectifs} :
\begin{itemize}
    \item Intégration complète de tous les modules
    \item Tests fonctionnels et d'intégration
    \item Tests automatisés avec Selenium
    \item Tests de performance avec JMeter
    \item Analyse de la qualité du code avec SonarQube
    \item Correction des anomalies identifiées
    \item Optimisation des performances
\end{itemize}

\textbf{Livrables} :
\begin{itemize}
    \item Système intégré et fonctionnel
    \item Rapports de tests (Selenium, JMeter)
    \item Rapport d'analyse SonarQube
    \item Documentation technique mise à jour
\end{itemize}

\subsection{Sprint 8 : Finalisation et documentation (Semaine 8)}

\textbf{Durée} : 1 semaine

\textbf{Objectifs} :
\begin{itemize}
    \item Finalisation de toutes les fonctionnalités
    \item Rédaction du rapport technique complet
    \item Préparation de la documentation utilisateur
    \item Préparation de la soutenance
    \item Validation finale avec les encadrants
\end{itemize}

\textbf{Livrables} :
\begin{itemize}
    \item Application finale déployée et opérationnelle
    \item Rapport technique complet
    \item Documentation utilisateur
    \item Présentation pour la soutenance
    \item Tous les livrables demandés dans le cahier des charges
\end{itemize}

Cette planification permet de structurer le travail de manière progressive, en s'assurant que chaque phase est complétée avant de passer à la suivante, tout en maintenant la flexibilité nécessaire pour s'adapter aux imprévus et aux retours des encadrants.

\section{Conclusion}

Ce premier chapitre a permis de présenter le contexte général du projet de formation professionnelle interactive avec coach virtuel. Nous avons identifié les limitations de la formation traditionnelle et défini les objectifs visant à offrir une expérience d'apprentissage personnalisée et dynamique. La solution proposée, basée sur une architecture moderne combinant Flutter, Spring Boot et l'IA générative, répond efficacement à ces défis. L'organisation du travail selon la méthodologie SCRUM sur 8 semaines, réparties en sprints itératifs, garantit une approche structurée et flexible pour la réalisation du projet. Le chapitre suivant présentera l'analyse et la conception détaillée du système à travers la modélisation UML.

% CHAPITRE 2
\chapter{ANALYSE ET CONCEPTION}
\setcounter{page}{8}

\section{Introduction}

Ce chapitre présente l'analyse et la conception du système de formation professionnelle interactive avec coach virtuel basé sur l'IA générative. L'objectif est de modéliser l'architecture du système, les interactions entre les différents acteurs et les fonctionnalités principales. Cette phase de conception permet de définir la structure du système avant son implémentation, en s'assurant que tous les besoins fonctionnels identifiés dans le cahier des charges sont pris en compte.

La modélisation proposée couvre les aspects fonctionnels du système, notamment la gestion des utilisateurs, l'interaction avec le coach virtuel, la personnalisation des parcours d'apprentissage, la génération automatique de contenu pédagogique, ainsi que le suivi et la supervision par les formateurs.

\section{Langage de modélisation}

Pour modéliser le système, nous utilisons le langage UML (Unified Modeling Language), qui est le standard de facto pour la modélisation orientée objet. UML permet de représenter différents aspects du système à travers plusieurs types de diagrammes :

\begin{itemize}
    \item \textbf{Diagrammes de cas d'utilisation} : pour identifier les fonctionnalités du système et les interactions entre les acteurs et le système.
    \item \textbf{Diagrammes de séquence} : pour modéliser les interactions temporelles entre les différents composants du système lors de l'exécution d'un cas d'utilisation.
    \item \textbf{Diagrammes de classes} : pour représenter la structure statique du système, les entités, leurs attributs, leurs méthodes et les relations entre elles.
\end{itemize}

Ces diagrammes fournissent une vision complète du système, depuis les besoins fonctionnels jusqu'à l'architecture technique, facilitant ainsi la communication entre les membres de l'équipe et les encadrants, ainsi que la transition vers la phase d'implémentation.

\section{Diagramme de cas d'utilisation général}

Le diagramme de cas d'utilisation général présente l'ensemble des fonctionnalités du système et les différents acteurs qui interagissent avec celui-ci. Les acteurs identifiés sont :

\begin{itemize}
    \item \textbf{Apprenant} : utilisateur principal du système, qui suit une formation et interagit avec le coach virtuel.
    \item \textbf{Formateur} : supervise les apprenants, consulte les statistiques et peut ajouter ou modifier du contenu pédagogique.
    \item \textbf{Administrateur} : gère les utilisateurs, les rôles et les permissions du système.
\end{itemize}

Les principaux cas d'utilisation identifiés sont :

\begin{itemize}
    \item Gestion de l'authentification et de l'inscription
    \item Gestion du profil utilisateur
    \item Interaction avec le coach virtuel (dialogue en langage naturel)
    \item Consultation du parcours d'apprentissage personnalisé
    \item Génération automatique de contenu pédagogique (quiz, exercices, résumés)
    \item Suivi de la progression et des statistiques
    \item Consultation du tableau de bord (pour formateurs)
    \item Gestion des contenus pédagogiques (pour formateurs)
    \item Gestion des utilisateurs et des permissions (pour administrateurs)
    \item Réception de notifications et d'alertes
\end{itemize}

\begin{figure}[h]
    \centering
    \includegraphics[width=0.9\textwidth]{images/diagramme_cas_utilisation.png}
    \caption{Diagramme de cas d'utilisation général du système}
    \label{fig:cas_utilisation}
\end{figure}

\section{Diagrammes de séquence}

Les diagrammes de séquence permettent de visualiser les interactions entre les différents composants du système lors de l'exécution d'un cas d'utilisation. Ils montrent l'ordre chronologique des messages échangés entre les acteurs et les différents modules (interface utilisateur, API backend, service IA, base de données).

\subsection{DS pour le cas d'utilisation : Authentification et connexion}

Ce diagramme de séquence illustre le processus d'authentification d'un utilisateur (apprenant, formateur ou administrateur) dans le système. Il montre les interactions entre l'interface Flutter, l'API Spring Boot, le service d'authentification et la base de données.

\begin{figure}[h]
    \centering
    \includegraphics[width=0.9\textwidth]{images/ds_authentification.png}
    \caption{Diagramme de séquence : Authentification et connexion}
    \label{fig:ds_auth}
\end{figure}

Le processus commence par la saisie des identifiants par l'utilisateur. L'interface Flutter envoie une requête à l'API backend qui vérifie les credentials auprès de la base de données. Si l'authentification réussit, un token JWT est généré et renvoyé au client, permettant ainsi l'accès aux fonctionnalités selon le rôle de l'utilisateur.

\subsection{DS pour le cas d'utilisation : Interaction avec le coach virtuel}

Ce diagramme de séquence modélise l'interaction entre un apprenant et le coach virtuel. L'apprenant pose une question en langage naturel, le système traite la requête via le module IA (LLM avec RAG), génère une réponse personnalisée et l'affiche à l'utilisateur. L'historique de la conversation est sauvegardé pour permettre un apprentissage continu.

\begin{figure}[h]
    \centering
    \includegraphics[width=0.9\textwidth]{images/ds_coach_virtuel.png}
    \caption{Diagramme de séquence : Interaction avec le coach virtuel}
    \label{fig:ds_coach}
\end{figure}

Le flux comprend plusieurs étapes : réception de la question, enrichissement du contexte via RAG (Recherche Augmentée par Génération), appel au modèle LLM pour générer la réponse, sauvegarde de l'échange dans l'historique, et retour de la réponse à l'utilisateur.

\subsection{DS pour le cas d'utilisation : Génération automatique de contenu pédagogique}

Ce diagramme de séquence représente le processus de génération automatique de contenu pédagogique (quiz, exercices, résumés) par le coach virtuel. Le système analyse le niveau et les besoins de l'apprenant, génère du contenu adapté via l'IA, et l'intègre dans le parcours d'apprentissage.

\begin{figure}[h]
    \centering
    \includegraphics[width=0.9\textwidth]{images/ds_generation_contenu.png}
    \caption{Diagramme de séquence : Génération automatique de contenu pédagogique}
    \label{fig:ds_generation}
\end{figure}

Le processus implique la récupération du profil et de la progression de l'apprenant, la génération de contenu personnalisé par le module IA, la validation éventuelle par un formateur (selon les paramètres), et l'ajout du contenu au parcours de l'apprenant.

\subsection{DS pour le cas d'utilisation : Consultation du tableau de bord et suivi}

Ce diagramme de séquence illustre la consultation du tableau de bord par un formateur pour suivre la progression des apprenants. Le système agrège les données de progression, calcule les statistiques et présente une vue d'ensemble des performances.

\begin{figure}[h]
    \centering
    \includegraphics[width=0.9\textwidth]{images/ds_tableau_bord.png}
    \caption{Diagramme de séquence : Consultation du tableau de bord}
    \label{fig:ds_dashboard}
\end{figure}

Le formateur demande l'affichage du tableau de bord, le backend récupère les données de progression depuis la base de données, calcule les statistiques (taux de complétion, notes moyennes, temps d'étude), et renvoie ces informations à l'interface pour visualisation.

\subsection{DS pour le cas d'utilisation : Personnalisation du parcours d'apprentissage}

Ce diagramme de séquence modélise l'adaptation dynamique du parcours d'apprentissage en fonction des performances de l'apprenant. Le système analyse les résultats, identifie les points forts et les difficultés, et ajuste automatiquement le parcours.

\begin{figure}[h]
    \centering
    \includegraphics[width=0.9\textwidth]{images/ds_personnalisation.png}
    \caption{Diagramme de séquence : Personnalisation du parcours d'apprentissage}
    \label{fig:ds_personnalisation}
\end{figure}

Le processus comprend l'analyse des performances récentes, l'évaluation du niveau de maîtrise des différents modules, la génération de recommandations personnalisées par l'IA, et la mise à jour du parcours de l'apprenant avec les nouveaux contenus suggérés.

\section{Diagramme de classes}

Le diagramme de classes représente la structure statique du système, en définissant les principales entités, leurs attributs, leurs méthodes et les relations entre elles. Ce diagramme constitue la base pour l'implémentation de la couche modèle dans le backend Spring Boot.

Les principales classes identifiées sont :

\begin{itemize}
    \item \textbf{User} : représente un utilisateur du système (apprenant, formateur ou administrateur) avec ses informations personnelles et ses préférences.
    \item \textbf{Course/Module} : représente un module de formation avec son contenu, ses objectifs et ses prérequis.
    \item \textbf{Progress} : suit la progression d'un apprenant dans un module (pourcentage de complétion, notes, temps passé).
    \item \textbf{Conversation} : stocke l'historique des échanges entre un apprenant et le coach virtuel.
    \item \textbf{Message} : représente un message dans une conversation (question ou réponse).
    \item \textbf{Content} : représente un contenu pédagogique généré automatiquement (quiz, exercice, résumé).
    \item \textbf{Notification} : représente une notification envoyée à un utilisateur.
    \item \textbf{Statistics} : agrège les statistiques de progression pour un apprenant ou un groupe.
\end{itemize}

\begin{figure}[h]
    \centering
    \includegraphics[width=0.95\textwidth]{images/diagramme_classes.png}
    \caption{Diagramme de classes du système}
    \label{fig:classes}
\end{figure}

Les relations principales incluent :
\begin{itemize}
    \item Un utilisateur peut avoir plusieurs progressions (un pour chaque module)
    \item Un utilisateur peut avoir plusieurs conversations avec le coach virtuel
    \item Une conversation contient plusieurs messages
    \item Un module peut contenir plusieurs contenus pédagogiques
    \item Un utilisateur peut recevoir plusieurs notifications
\end{itemize}

\section{Conclusion}

Ce chapitre a présenté l'analyse et la conception du système à travers la modélisation UML, incluant les diagrammes de cas d'utilisation, de séquence et de classes. Cette modélisation permet de clarifier les besoins fonctionnels, de définir la structure des données et de servir de référence pour la phase d'implémentation. La base de conception établie couvre tous les aspects essentiels du système, depuis la gestion des utilisateurs jusqu'à l'interaction avec le coach virtuel basé sur l'IA générative. Le chapitre suivant détaillera les technologies et outils techniques choisis pour concrétiser cette conception.

% CHAPITRE 3
\chapter{TECHNOLOGIES ET OUTILS UTILISES}
\setcounter{page}{18}

\section{Introduction}

Ce chapitre présente l'ensemble des technologies, frameworks et outils utilisés pour le développement de la plateforme de formation professionnelle interactive. Le choix de ces technologies a été guidé par plusieurs critères : la performance, la maintenabilité, la scalabilité, la compatibilité avec les objectifs du projet et l'expertise de l'équipe.

\section{Technologies utilisées}

\subsection{Front-end : Flutter}

Flutter est un framework open-source développé par Google pour créer des applications multiplateformes avec une seule base de code. Il permet de développer des applications pour iOS, Android, Web et Desktop.

\textbf{Avantages de Flutter pour ce projet} :
\begin{itemize}
    \item Développement multiplateforme avec une seule base de code
    \item Performance native grâce à la compilation en code machine
    \item Interface utilisateur riche et personnalisable
    \item Hot reload pour un développement rapide
    \item Grande communauté et documentation complète
\end{itemize}

\subsection{Back-end : Spring Boot}

Spring Boot est un framework Java qui simplifie le développement d'applications Java en fournissant une configuration automatique et une infrastructure prête à l'emploi.

\textbf{Avantages de Spring Boot pour ce projet} :
\begin{itemize}
    \item Architecture en couches claire (Controller, Service, Repository)
    \item Intégration facile avec les bases de données (JPA/Hibernate)
    \item Sécurité intégrée (Spring Security)
    \item Gestion des API REST simplifiée
    \item Écosystème Spring riche et mature
\end{itemize}

\subsection{Intelligence Artificielle : LLM et RAG}

Pour le module de coach virtuel, nous utilisons une approche combinant un modèle de langage (LLM) avec la technique RAG (Retrieval-Augmented Generation).

\textbf{LLM (Large Language Model)} :
\begin{itemize}
    \item Modèle de langage capable de comprendre et générer du texte en langage naturel
    \item Permet des conversations contextuelles avec les apprenants
    \item Génération de contenu pédagogique adapté
\end{itemize}

\textbf{RAG (Retrieval-Augmented Generation)} :
\begin{itemize}
    \item Enrichit les réponses du LLM avec des informations provenant de la base de connaissances
    \item Améliore la précision et la pertinence des réponses
    \item Réduit les hallucinations du modèle
\end{itemize}

\section{Outils de base de données : MySQL/PostgreSQL}

Le choix entre MySQL et PostgreSQL dépend des besoins spécifiques du projet. Les deux sont des systèmes de gestion de bases de données relationnelles robustes et largement utilisés.

\textbf{Caractéristiques} :
\begin{itemize}
    \item Support des transactions ACID
    \item Gestion efficace des relations entre entités
    \item Performance optimisée pour les requêtes complexes
    \item Compatibilité avec JPA/Hibernate
\end{itemize}

\section{Environnement de travail}

\subsection{Outils de développement}

\begin{itemize}
    \item \textbf{IDE} : IntelliJ IDEA pour le développement Spring Boot, VS Code ou Android Studio pour Flutter
    \item \textbf{Versioning} : Git avec GitHub pour le contrôle de version et la collaboration
    \item \textbf{API Testing} : Postman pour tester les API REST
\end{itemize}

\subsection{Outils de qualité et tests}

\begin{itemize}
    \item \textbf{SonarQube} : Analyse de la qualité du code (maintenabilité, sécurité, couverture de tests)
    \item \textbf{Selenium} : Tests automatisés de l'interface utilisateur
    \item \textbf{JMeter} : Tests de performance et de charge
    \item \textbf{JUnit} : Framework de tests unitaires pour Java
\end{itemize}

\section{Outils de gestion de projet}

\begin{itemize}
    \item \textbf{Trello/Jira/Azure DevOps} : Suivi des tâches et gestion des sprints
    \item \textbf{GitHub} : Gestion du code source et collaboration
    \item \textbf{Documentation} : Markdown pour la documentation technique
\end{itemize}

\section{Conclusion}

Ce chapitre a présenté l'ensemble des technologies et outils choisis pour le développement de la plateforme, incluant Flutter pour le front-end, Spring Boot pour le back-end, et les modules d'IA (LLM + RAG) pour le coach virtuel. Le choix de ces technologies modernes et performantes garantit une solution scalable, maintenable et adaptée aux besoins du projet. Le chapitre suivant détaillera l'implémentation concrète de ces technologies et la mise en œuvre des différentes interfaces du système.

% CHAPITRE 4
\chapter{IMPLEMENTATION ET MISE EN OEUVRE}
\setcounter{page}{24}

\section{Introduction}

Ce chapitre présente l'implémentation concrète de la plateforme de formation professionnelle interactive. Il décrit l'architecture logicielle mise en place, les différentes interfaces développées et les choix techniques effectués lors de la réalisation.

\section{Architecture logicielle}

L'architecture du système suit une approche en couches, séparant clairement les responsabilités entre le front-end, le back-end et la base de données. Cette séparation permet une maintenabilité optimale et une évolutivité facilitée.

\subsection{Architecture backend}

Le backend Spring Boot suit une architecture en couches :

\begin{itemize}
    \item \textbf{Couche Controller} : Gère les requêtes HTTP et la sérialisation/désérialisation des données
    \item \textbf{Couche Service} : Contient la logique métier et orchestre les opérations
    \item \textbf{Couche Repository} : Gère l'accès aux données via JPA/Hibernate
    \item \textbf{Couche Entity} : Modèles de données correspondant aux tables de la base de données
\end{itemize}

\subsection{Architecture frontend}

L'application Flutter suit une architecture modulaire :

\begin{itemize}
    \item \textbf{Pages/Screens} : Interfaces utilisateur
    \item \textbf{Widgets} : Composants réutilisables
    \item \textbf{Services} : Gestion des appels API et de la logique métier
    \item \textbf{Models} : Modèles de données côté client
    \item \textbf{State Management} : Gestion de l'état de l'application (Provider, Bloc, ou Riverpod)
\end{itemize}

\subsection{Intégration IA}

Le module d'intelligence artificielle est intégré au backend via des services dédiés :

\begin{itemize}
    \item Service de génération de réponses (LLM)
    \item Service de recherche et récupération de contexte (RAG)
    \item Service de génération de contenu pédagogique
\end{itemize}

\section{Les interfaces web}

L'application web développée avec Flutter offre une expérience utilisateur complète et responsive. Les principales interfaces incluent :

\subsection{Interface d'authentification}

\begin{itemize}
    \item Page de connexion avec validation des identifiants
    \item Page d'inscription avec gestion des rôles
    \item Gestion des sessions et tokens JWT
\end{itemize}

\subsection{Interface apprenant}

\begin{itemize}
    \item Tableau de bord personnel avec progression
    \item Interface de chat avec le coach virtuel
    \item Consultation du parcours d'apprentissage personnalisé
    \item Accès aux contenus pédagogiques générés
    \item Historique des conversations et activités
\end{itemize}

\subsection{Interface formateur}

\begin{itemize}
    \item Tableau de bord avec statistiques des apprenants
    \item Visualisation de la progression individuelle et collective
    \item Gestion des contenus pédagogiques
    \item Validation des contenus générés par l'IA
    \item Système d'alertes et de notifications
\end{itemize}

\subsection{Interface administrateur}

\begin{itemize}
    \item Gestion des utilisateurs et des rôles
    \item Configuration du système
    \item Gestion des permissions
    \item Monitoring et logs
\end{itemize}

\section{Les interfaces mobiles}

L'application mobile, également développée avec Flutter, offre les mêmes fonctionnalités que la version web, optimisées pour les écrans tactiles et la mobilité.

\subsection{Adaptations mobiles}

\begin{itemize}
    \item Interface responsive adaptée aux différentes tailles d'écran
    \item Navigation optimisée pour le tactile
    \item Notifications push pour les rappels et alertes
    \item Mode hors ligne pour la consultation de contenu
    \item Synchronisation automatique des données
\end{itemize}

\subsection{Fonctionnalités spécifiques mobile}

\begin{itemize}
    \item Accès rapide au coach virtuel depuis l'écran d'accueil
    \item Widgets pour afficher la progression
    \item Partage de contenu et de progression
    \item Intégration avec le calendrier pour les sessions
\end{itemize}

\section{Conclusion}

L'implémentation de la plateforme a permis de concrétiser les objectifs définis dans le cahier des charges à travers une architecture en couches bien structurée. Les interfaces développées, tant pour le web que pour le mobile, répondent efficacement aux besoins des différents types d'utilisateurs (apprenants, formateurs, administrateurs). Cette implémentation offre une base solide pour l'évolution future du système et démontre la faisabilité technique de la solution proposée.

% Conclusion générale
\chapter*{Conclusion générale}
\addcontentsline{toc}{chapter}{Conclusion générale}

Ce projet de développement d'une plateforme de formation professionnelle interactive avec coach virtuel basé sur l'IA générative a permis de répondre aux défis identifiés dans la formation traditionnelle. La solution proposée combine une architecture technique moderne (Flutter, Spring Boot, IA) avec une approche méthodologique agile (SCRUM) pour offrir une expérience d'apprentissage personnalisée et dynamique.

Les objectifs fixés ont été atteints : développement d'un système de dialogue intelligent, personnalisation des parcours, génération automatique de contenu, suivi efficace par les formateurs, et accessibilité multiplateforme. La modélisation UML réalisée a servi de base solide pour l'implémentation, tandis que les technologies choisies ont permis de développer une solution scalable et maintenable.

Les perspectives d'évolution incluent l'amélioration continue du modèle d'IA, l'ajout de nouvelles fonctionnalités pédagogiques, et l'extension à d'autres domaines de formation.

% Références
\chapter*{Références}
\addcontentsline{toc}{chapter}{Références}

\begin{thebibliography}{99}
    \bibitem{flutter} Flutter Documentation. \textit{https://flutter.dev/docs}
    
    \bibitem{spring} Spring Boot Documentation. \textit{https://spring.io/projects/spring-boot}
    
    \bibitem{uml} Object Management Group. \textit{UML Specification}. \textit{https://www.omg.org/spec/UML/}
    
    \bibitem{scrum} Schwaber, K., \& Sutherland, J. (2020). \textit{The Scrum Guide}. \textit{https://scrumguides.org/}
    
    \bibitem{rag} Lewis, P., et al. (2020). \textit{Retrieval-Augmented Generation for Knowledge-Intensive NLP Tasks}. NeurIPS.
    
    \bibitem{llm} Brown, T., et al. (2020). \textit{Language Models are Few-Shot Learners}. NeurIPS.
\end{thebibliography}

\end{document}

